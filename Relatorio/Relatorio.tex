\documentclass{article}
\usepackage[utf8]{inputenc}
\usepackage{amsmath}
\usepackage{amssymb}
\usepackage{graphicx}
\usepackage{tikz}
\usepackage{enumitem}

\title{SME0211 - Otimização Linear\\
Segundo semestre de 2024\\ 

\textbf{\\Trabalho final}}

\author{
Katlyn Ribeiro Almeida -- 14586070\\
Ian de Holanda Cavalcanti Bezerra -- 13835412 \\
Julia Graziosi Ortiz -- 11797810\\
Cody Stefano Barham Setti -- 4856322\\
Matheus Araujo Pinheiro -- 14676810
}

\begin{document}
\maketitle
%-----------------------------------------------------------------------------------------
\section{Para Fazer}

\begin{itemize}
    \item Transformar o problema em um problema na forma padrao
    \item Determinar A, b e c com os zeros nos indices certos e com variaveis de folga (da forma padrao)
    \item Rezar
    \item Rodar o codigo
    \item Relatorio - Capitulo Simplex
    \item Relatorio - Capitulo Mostrar codigo
    \item Relatorio - Capitulo Apresentar problema e modelagem
    \item Relatorio - Capitulo Apresentar problemas encontrados na solucao
    \item Relatorio - Capitulo Apresentar resultados e conclusoes
\end{itemize}






\section{Escolha de ferramentas}

Para o desenvolvimento deste projeto, optamos por utilizar a linguagem de programação Python, com ênfase especial nos notebooks Jupyter. Esta escolha foi motivada por diversas razões:

\begin{itemize}
    \item \textbf{Facilidade na implementação de algoritmos iterativos:} Os notebooks Jupyter oferecem um ambiente interativo que é particularmente adequado para a implementação e teste de algoritmos que requerem múltiplas iterações.
    
    \item \textbf{Ambiente de execução flexível:} No ambiente do notebook, podemos inicializar variáveis e realizar operações sobre elas sem a necessidade de reinicializá-las a cada execução. Isso proporciona uma grande flexibilidade no desenvolvimento e depuração do código.
    
    \item \textbf{Visualização integrada:} Os notebooks Jupyter permitem a integração de código, resultados e visualizações, facilitando a análise e apresentação dos resultados obtidos.
    
    \item \textbf{Simplicidade e eficiência:} Python oferece uma sintaxe clara e intuitiva, facilitando a implementação de algoritmos complexos. Além disso, suas bibliotecas, como NumPy e SciPy, fornecem funções otimizadas para a solução de sistemas lineares, permitindo uma implementação eficiente e concisa do método Simplex.
\end{itemize}


%-----------------------------------------------------------------------------------------
\section{Otimização/Programação Linear}

A otimização linear é uma técnica matemática que busca encontrar o valor máximo ou mínimo de uma função linear, sujeita a um conjunto de restrições lineares. Este problema é representado por uma função objetivo linear e um conjunto de desigualdades lineares que limitam as soluções possíveis.

Esses problemas são amplamente utilizados em várias áreas, como economia, logística, produção e finanças, para maximizar lucros, minimizar custos ou otimizar a utilização de recursos.


Para facilitar e unificar as formas de solução desses problemas, buscamos representá-los na forma padrão, resolvendo um sistema de minimização sujeito a restrições de igualdades. Essa forma padrão é frequentemente obtida através da adição de variáveis de folga nas desigualdades originais.

A forma padrão de um problema de programação linear pode ser expressa da seguinte maneira:

\begin{align*}
\text{Minimizar } & c^T x \\
\text{Sujeito a } & Ax = b \\
& x \geq 0
\end{align*}

Onde:
\begin{itemize}
    \item $x$ é o vetor de variáveis de decisão
    \item $c$ é o vetor de coeficientes da função objetivo
    \item $A$ é a matriz de coeficientes das restrições
    \item $b$ é o vetor de termos independentes das restrições
\end{itemize}

Esta representação padronizada permite unificar as aplicaçoes de métodos de solução, como o algoritmo Simplex, a uma ampla variedade de problemas de otimização linear, independentemente de sua formulação original.


\begin{equation}
\begin{aligned}
f(\boldsymbol{x}) &=  1,6A_1 + 2,4AI_1 + 3F_1 + 2S_1 + 7PA_1 + 4PV_1 + 3,6G_1 \\
& + 0,8A_2 + 1,2AI_2 + 1,8F_2 + 1S_2 + 4PA_2 + 2PV_2 + 1,5G_2 \\
& + 1,6A_3 + 2,4AI_3 + 3F_3 + 2S_3 + 10PA_3 + 5PV_3 + 3,6G_3 \\
& + 0,8A_4 + 1,2AI_4 + 1,5F_4 + 1S_4 + 3,5PA_4 + 1PV_4 + 2,4G_4
\end{aligned}
\end{equation}

\begin{equation}
\begin{tabular}{ccccccccc}
c = [& 1.6 & 2.4 & 3 & 2 & 7 & 4 & 3.6 &  \\
& 0.8 & 1.2 & 1.8 & 1 & 4 & 2 & 1.5 &  \\
& 1.6 & 2.4 & 3 & 2 & 10 & 5 & 3.6 &  \\
& 0.8 & 1.2 & 1.5 & 1 & 3.5 & 1 & 2.4 & ]
\end{tabular}
\end{equation}

As restrições referentes à primeira refeição são:
\begin{align*}
& 128.6A1 + 112AI1 + 76F1 + 25S1 + 250PA1 + 9.5G1 \leq 800 \text{ (cal)} \\
& 128.6A1 + 112AI1 + 76F1 + 25S1 + 155PV1 + 9.5G1 \leq 800 \\
& 2.69A1 + 2.32AI1 + 4.3F1 + 1.5S1 + 25PA1 + 2.5G1 \geq 60 \text{ (prot)} \\
& 2.69A1 + 2.32AI1 + 4.3F1 + 1.5S1 + 11.1PV1 + 2.5G1 \geq 60 \\
& 27.9A1 + 23.51AI1 + 14F1 + 5S1 + 15PA1 + 18.6G1 \geq 120 \text{ (carb)} \\
& 27.9A1 + 23.51AI1 + 14F1 + 5S1 + 2.2PV1 + 18.6G1 \geq 120
\end{align*}

As restrições referentes à segunda refeição são:
\begin{align*}
& 128.6A2 + 112AI2 + 76F2 + 25S2 + 300PA2 + 400G2 \leq 800 \text{ (cal)} \\
& 128.6A2 + 112AI2 + 76F2 + 25S2 + 120PV2 + 400G2 \leq 800 \\
& 2.69A2 + 2.32AI2 + 4.3F2 + 1.5S2 + 15PA2 + 1.2G2 \geq 60 \text{ (prot)} \\
& 2.69A2 + 2.32AI2 + 4.3F2 + 1.5S2 + 3PV2 + 1.2G2 \geq 60 \\
& 27.9A2 + 23.51AI2 + 14F2 + 5S2 + 2PA2 + 85.1G2 \geq 120 \text{ (carb)} \\
& 27.9A2 + 23.51AI2 + 14F2 + 5S2 + 20PV2 + 85.1G2 \geq 120
\end{align*}

As restrições referentes à terceira refeição são:
\begin{align*}
& 128.6A3 + 112AI3 + 76F3 + 25S3 + 250PA3 + 120G3 \leq 800 \text{ (cal)} \\
& 128.6A3 + 112AI3 + 76F3 + 25S3 + 180PV3 + 120G3 \leq 800 \\
& 2.69A3 + 2.32AI3 + 4.3F3 + 1.5S3 + 25PA3 + 2G3 \geq 60 \text{ (prot)} \\
& 2.69A3 + 2.32AI3 + 4.3F3 + 1.5S3 + 15PV3 + 2G3 \geq 60 \\
& 27.9A3 + 23.51AI3 + 14F3 + 5S3 + 5PA3 + 25G3 \geq 120 \text{ (carb)} \\
& 27.9A3 + 23.51AI3 + 14F3 + 5S3 + 25PV3 + 25G3 \geq 120
\end{align*}


As restrições referentes à quarta refeição são:
\begin{align*}
& 128.6A4 + 112AI4 + 76F4 + 25S4 + 200PA4 + 157G4 \leq 800 \text{ (cal)} \\
& 128.6A4 + 112AI4 + 76F4 + 25S4 + 150PV4 + 157G4 \leq 800 \\
& 2.69A4 + 2.32AI4 + 4.3F4 + 1.5S4 + 25PA4 + 5.8G4 \geq 60 \text{ (prot)} \\
& 2.69A4 + 2.32AI4 + 4.3F4 + 1.5S4 + 13PV4 + 5.8G4 \geq 60 \\
& 27.9A4 + 23.51AI4 + 14F4 + 5S4 + 5PA4 + 30.9G4 \geq 120 \text{ (carb)} \\
& 27.9A4 + 23.51AI4 + 14F4 + 5S4 + 1PV4 + 30.9G4 \geq 120
\end{align*}

%-----------------------------------------------------------------------------------------
\section{O Algoritmo Simplex}







\end{document}
