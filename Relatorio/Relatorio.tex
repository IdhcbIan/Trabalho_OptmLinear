\documentclass{article}
\usepackage[utf8]{inputenc}
\usepackage{amsmath}
\usepackage{amssymb}
\usepackage{graphicx}
\usepackage{tikz}
\usepackage{enumitem}

\title{SME0211 - Otimização Linear\\
Segundo semestre de 2024\\ 

\textbf{\\Trabalho final}}

\author{
Katlyn Ribeiro Almeida - 14586070\\
Ian de Holanda Cavalcanti Bezerra - 13835412 \\
Julia Graziosi Ortiz - 11797810\\
Matheus Araujo Pinheiro - 14676810
}

\begin{document}
\maketitle
%-----------------------------------------------------------------------------------------
\section{Escolha de ferramentas}

Para o desenvolvimento deste projeto, optamos por utilizar a linguagem de programação Python, com ênfase especial nos notebooks Jupyter. Esta escolha foi motivada por diversas razões:

\begin{itemize}
    \item \textbf{Facilidade na implementação de algoritmos iterativos:} Os notebooks Jupyter oferecem um ambiente interativo que é particularmente adequado para a implementação e teste de algoritmos que requerem múltiplas iterações.
    
    \item \textbf{Ambiente de execução flexível:} No ambiente do notebook, podemos inicializar variáveis e realizar operações sobre elas sem a necessidade de reinicializá-las a cada execução. Isso proporciona uma grande flexibilidade no desenvolvimento e depuração do código.
    
    \item \textbf{Visualização integrada:} Os notebooks Jupyter permitem a integração de código, resultados e visualizações, facilitando a análise e apresentação dos resultados obtidos.
    
    \item \textbf{Simplicidade e eficiência:} Python oferece uma sintaxe clara e intuitiva, facilitando a implementação de algoritmos complexos. Além disso, suas bibliotecas, como NumPy e SciPy, fornecem funções otimizadas para a solução de sistemas lineares, permitindo uma implementação eficiente e concisa do método Simplex.
\end{itemize}


%-----------------------------------------------------------------------------------------
\section{Otimização/Programação Linear}

A otimização linear é uma técnica matemática que busca encontrar o valor máximo ou mínimo de uma função linear, sujeita a um conjunto de restrições lineares. Este problema é representado por uma função objetivo linear e um conjunto de desigualdades lineares que limitam as soluções possíveis.

Esses problemas são amplamente utilizados em várias áreas, como economia, logística, produção e finanças, para maximizar lucros, minimizar custos ou otimizar a utilização de recursos.


Para facilitar e unificar as formas de solução desses problemas, buscamos representá-los na forma padrão, resolvendo um sistema de minimização sujeito a restrições de igualdades. Essa forma padrão é frequentemente obtida através da adição de variáveis de folga nas desigualdades originais.

A forma padrão de um problema de programação linear pode ser expressa da seguinte maneira:

\begin{align*}
\text{Minimizar } & c^T x \\
\text{Sujeito a } & Ax = b \\
& x \geq 0
\end{align*}

Onde:
\begin{itemize}
    \item $x$ é o vetor de variáveis de decisão
    \item $c$ é o vetor de coeficientes da função objetivo
    \item $A$ é a matriz de coeficientes das restrições
    \item $b$ é o vetor de termos independentes das restrições
\end{itemize}

Esta representação padronizada permite unificar as aplicaçoes de métodos de solução, como o algoritmo Simplex, a uma ampla variedade de problemas de otimização linear, independentemente de sua formulação original.



%-----------------------------------------------------------------------------------------
\section{O Algoritmo Simplex}







\end{document}
