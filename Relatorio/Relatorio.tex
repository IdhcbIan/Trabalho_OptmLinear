\documentclass{article}
\usepackage[utf8]{inputenc}
\usepackage{amsmath}
\usepackage{amssymb}
\usepackage{graphicx}
\usepackage{tikz}
\usepackage{enumitem}

\title{SME0211 - Otimização Linear\\
Segundo semestre de 2024\\ 
Lista de exercícios 9}
\author{Ian Bezerra - 13835412}

\begin{document}

\maketitle

\section*{Problema 1}

\

Considere o poliedro $P = \{x \mid Ax \geq b\}$, com

$$
A = \begin{pmatrix}
1 & 0 & 2 \\
5 & -2 & 6 \\
-1 & 2 & 0 \\
2 & -1 & 1 \\
0 & 1 & 1
\end{pmatrix}
\quad \text{e} \quad
b = \begin{pmatrix}
0 \\
5 \\
1 \\
2 \\
1
\end{pmatrix}
$$

\begin{enumerate}
\item Dado o ponto $x = (1,1,1)$ e o vetor direção $d = (3,2,1)$, a reta $y = x + \lambda d$, com $\lambda \in \mathbb{R}$, está inteiramente dentro do poliedro $P$? Justifique.

\item O poliedro $P$ tem algum ponto extremo? Justifique.
\end{enumerate}

\section*{Solução do item 1}

Para determinar se a reta $y = x + \lambda d$ está inteiramente dentro do poliedro $P$, precisamos verificar se todos os pontos nesta reta satisfazem as desigualdades que definem $P$ para todos os valores de $\lambda$.

\textbf{Substituindo a equação da reta nas desigualdades do poliedro}

Precisamos verificar se $A(x + \lambda d) \geq b$ para todo $\lambda \in \mathbb{R}$.

Calculemos $A(x + \lambda d)$:

$$
A(x + \lambda d) = A \begin{pmatrix}
1 + 3\lambda \\
1 + 2\lambda \\
1 + \lambda
\end{pmatrix}
$$

\textbf{Avaliando as desigualdades}

\begin{enumerate}
\item $1(1 + 3\lambda) + 0(1 + 2\lambda) + 2(1 + \lambda) \geq 0$
   $1 + 3\lambda + 2 + 2\lambda \geq 0$
   $3 + 5\lambda \geq 0$

\item $5(1 + 3\lambda) - 2(1 + 2\lambda) + 6(1 + \lambda) \geq 5$
   $5 + 15\lambda - 2 - 4\lambda + 6 + 6\lambda \geq 5$
   $9 + 17\lambda \geq 5$

\item $-1(1 + 3\lambda) + 2(1 + 2\lambda) + 0(1 + \lambda) \geq 1$
   $-1 - 3\lambda + 2 + 4\lambda \geq 1$
   $1 + \lambda \geq 1$

\item $2(1 + 3\lambda) - 1(1 + 2\lambda) + 1(1 + \lambda) \geq 2$
   $2 + 6\lambda - 1 - 2\lambda + 1 + \lambda \geq 2$
   $2 + 5\lambda \geq 2$

\item $0(1 + 3\lambda) + 1(1 + 2\lambda) + 1(1 + \lambda) \geq 1$
   $1 + 2\lambda + 1 + \lambda \geq 1$
   $2 + 3\lambda \geq 1$
\end{enumerate}

\textbf{Com isso temos que.}

Para que a reta esteja inteiramente dentro do poliedro, todas essas desigualdades devem ser satisfeitas para todo $\lambda \in \mathbb{R}$. Simplificando:

\begin{enumerate}
\item $\lambda \geq -\frac{3}{5}$
\item $\lambda \geq -\frac{4}{17}$
\item $\lambda \geq 0$
\item $\lambda \geq 0$
\item $\lambda \geq -\frac{1}{3}$
\end{enumerate}

\textbf{Portanto.}

Para que a reta esteja inteiramente dentro do poliedro, todas essas desigualdades devem ser satisfeitas simultaneamente para todo $\lambda \in \mathbb{R}$. No entanto, podemos ver que isso não é possível devido à terceira e quarta desigualdades, que exigem $\lambda \geq 0$.

Portanto, a reta $y = x + \lambda d$ não está inteiramente dentro do poliedro $P$. Apenas a parte da reta onde $\lambda \geq 0$ está dentro do poliedro.

\textbf{Expandindo.}

A reta intersecta a fronteira do poliedro em $\lambda = 0$, que corresponde ao ponto $x = (1,1,1)$. Para $\lambda < 0$, a reta está fora do poliedro, enquanto para $\lambda \geq 0$, ela permanece dentro. Isso significa que apenas um raio (semi-reta) partindo de $(1,1,1)$ na direção de $d = (3,2,1)$ está contido dentro do poliedro, não a reta inteira.

\section*{Solução do item 2}

Para determinar se o poliedro $P$ tem algum ponto extremo, vamos analisar a estrutura do sistema $Ax \geq b$.

\textbf{Análisando a da matriz $A$}

A matriz $A$ é:

$$
A = \begin{pmatrix}
1 & 0 & 2 \\
5 & -2 & 6 \\
-1 & 2 & 0 \\
2 & -1 & 1 \\
0 & 1 & 1
\end{pmatrix}
$$

\textbf{Verificar a direção de recessão}

Uma direção de recessão $d$ satisfaz $Ad \geq 0$. Se existir uma direção de recessão não-nula, o poliedro não terá pontos extremos.

Resolvendo $Ad \geq 0$:

\begin{align*}
d_1 + 2d_3 &\geq 0 \\
5d_1 - 2d_2 + 6d_3 &\geq 0 \\
-d_1 + 2d_2 &\geq 0 \\
2d_1 - d_2 + d_3 &\geq 0 \\
d_2 + d_3 &\geq 0
\end{align*}

\textbf{Análise as direções de recessão}

Observamos que $d = (0, 0, 1)$ satisfaz todas as desigualdades acima. Esta é uma direção de recessão não-nula.

\textbf{Conclusão.}

A existência de uma direção de recessão não-nula implica que o poliedro $P$ se estende infinitamente e não possui pontos extremos.

\end{document}
